The assignment is divided in several tasks. The first task is to build a tool chain for extracting information from a Java program. The tool should extract at least the following information: class names, (static) fields, method signatures, associations, generalisations, realisations and dependencies (defined in Table 3.1 of the book by Tonella and Potrich). The tool chain has to be build using Rascal. Rascal is a domain specific language for source code analysis and manipulation.

The second part is the architecture reconstruction. From the extracted information a class diagram has to be reconstructed. In this diagram there must be a taken care of declared vs. actual types and containers. This diagram has to be visualized.

Finally the tool chain has to be applied on two projects eLib and two versions of the CyberNeko HTML Parser. The acquired results has to be discussed and described in this report.

\section*{Purpose of the assignment}
The purpose of the assignment is identifying interrelations and components of an existing system. Good documentation of the system is required for cost effectively maintenance and modification of a system. Since the documentation of a system can be lacking or outdated, due to the evolution of a system. It is convenient to have tools that extract this information from the existing software (reverse engineering) and generate the documentation like class and flow diagrams directly from the code base. 